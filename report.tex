\documentclass[dvipsnames,pdf, unicode, 12pt, a4paper, oneside, fleqn]{article}
\usepackage[utf8]{inputenc}
\usepackage[T2B]{fontenc}
\usepackage[english,russian]{babel}

\usepackage{listings}
\usepackage{longtable}
\oddsidemargin=-0.4mm
\textwidth=160mm
\topmargin=4.6mm
\textheight=210mm
\usepackage{geometry}
%% Страницы диссертациия должны иметь следующие поля:
%% левое --- 25 мм, правое --- 10 мм, верхнее --- 20 мм, нижнее --- 20 мм.
%% Абзацный отступ должен быть одинаковым по всему тексту и равен пяти знакам.
\geometry{
 a4paper,
 total={170mm,257mm},
 right=10mm,
 left=10mm,
 top=20mm,
 bottom=20mm,
}
\pagenumbering{gobble}

\usepackage{multicol}
\usepackage[]{amsmath}
\usepackage{multirow}

% THIS IS MY NEWLY DEFINED COMMAND
\newcommand\tline[2]{$\underset{\text{#1}}{\text{\underline{\hspace{#2}}}}$}

\usepackage{csquotes}
\DeclareQuoteStyle{russian}
    {\guillemotleft}{\guillemotright}[0.025em]
    {\quotedblbase}{\textquotedblleft}
\ExecuteQuoteOptions{style=russian}

\usepackage{longtable,array}

\newcolumntype{C}[1]{>{\centering\arraybackslash}p{#1}}
\setlength{\extrarowheight}{10pt}

\begin{document}

\begin{titlepage}
\begin{center}
\bfseries{\Large Министерство образования и науки\\Российской Федерации}

\vspace{12pt}

\bfseries{\Large Московский авиационный институт\\ (национальный исследовательский университет)}

\vspace{48pt}


{\large Факультет информационных технологий и прикладной математики}

\vspace{36pt}


{\large Кафедра вычислительной математики и программирования}

\vspace{48pt}

{\huge ЖУРНАЛ}

\vspace{12pt}

{\large ПО ПРОИЗВОДСТВЕННОЙ ПРАКТИКЕ}


\end{center}

\vspace{72pt}

\begin{flushleft}
Наименование практики: {\itshape вычислительная}\\
Студентs: В.\,Д. Буркевич\, А.\,В. Шухова \\
Факультет №8, курс 2, группа 7 \\
\end{flushleft}

\vspace{12pt}

\begin{flushleft}
Практика с 29.06.19 по 12.07.19
\end{flushleft}

\vfill

\begin{center}
\bfseries Москва, \the\year
\end{center}
\end{titlepage}

\pagebreak

\begin{center}
\bfseries{\large ИНСТРУКЦИЯ }

\vspace{12pt}

\bfseries{о заполнении журнала по производственной практике}
\end{center}

\begin{multicols}{2}
{\small
Журнал по производственной практике студентов имеет единую форму для всех видов практик.

Задание в журнал вписывается руководителем практики от института в первые три-пять дней пребывания студентов на практике в соответствии с тематикой, утверждённой на кафедре до начала практики. Журнал по производственной практике является основным документом для текущего и итогового контроля выполнения заданий, требований инструкции и программы практики.

Табель прохождения практики, задание, а также технический отчёт выполняются каждым студентом самостоятельно.

Журнал заполняется студентом непрерывно в процессе прохождения всей практики и регулярно представляется для просмотра руководителям практики. Все их замечания подлежат немедленному выполнению.

В разделе «Табель прохождения практики» ежедневно должно быть указано, на каких рабочих местах и в качестве кого работал студент. Эти записи проверяются и заверяются цеховыми руководителями практики, в том числе мастерами и бригадирами. График прохождения практики заполняется в соответствии с графиком распределения студентов по рабочим местам практики, утверждённым руководителем предприятия.
В разделе «Рационализаторские предложения» должно быть приведено содержание поданных в цехе рационализаторских предложений со всеми необходимыми расчётами и эскизами. Рационализаторские предложения подаются индивидуально и коллективно.

Выполнение студентом задания по общественно-политической практике заносятся в раздел «Общественно-политическая практика». Выполнение работы по оказанию практической помощи предприятию (участие в выполнении спецзаданий, работа сверхурочно и т.п.) заносятся в раздел журнала «Работа в помощь предприятию» с последующим письменным подтверждением записанной работы соответствующими цеховыми руководителями.
Раздел «Технический отчёт по практике» должен быть заполнен особо тщательно. Записи необходимо делать чернилами в сжатой, но вместе с тем чёткой и ясной форме и технически грамотно. Студент обязан ежедневно подробно излагать содержание работы, выполняемой за каждый день. Содержание этого раздела должно отвечать тем конкретным требованиям, которые предъявляются к техническому отчёту заданием и программой практики. Технический отчёт должен показать умение студента критически оценивать работу данного производственного участка и отразить, в какой степени студент способен применить теоретические знания для решения конкретных производственных задач.

Иллюстративный и другие материалы, использованные студентом в других разделах журнала, в техническом отчёте не должны повторяться, следует ограничиваться лишь ссылкой на него. Участие студентов в производственно-технической конференции, выступление с докладами, рационализаторские предложения и т.п. должны заноситься на свободные страницы журнала.

{\bfseries Примечание.} Синьки, кальки и другие дополнения к журналу могут быть сделаны только с разрешения администрации предприятия и должны подшиваться в конце журнала.

Руководители практики от института обязаны следить за тем, чтобы каждый цеховой руководитель практики перед уходом студентов из данного цеха в другой цех вписывал в журнал студента отзывы об их работе в цехе.

Текущий контроль работы студентов осуществляется руководители практики от института и цеховыми руководителями практики заводов. Все замечания студентам руководители делают в письменном виде на страницах журнала, ставя при этом свою подпись и дату проверки.

Результаты защиты технического отчёта заносятся в протокол и одновременно заносятся в ведомость и зачётную книжку студента.

{\bfseries Примечание.} Нумерация чистых страниц журнала проставляется каждым студентом в своём журнале до начала практики.
}
\end{multicols}

\begin{center}
С инструкцией о заполнении журнала ознакомились:
\end{center}

«\hspace{0.5cm}» \tline{(дата)}{1.5in} \the\year\,г.\hfillСтудент Буркевич В.\,Д. \tline{(подпись)}{1in}

«\hspace{0.5cm}» \tline{(дата)}{1.5in} \the\year\,г.\hfillСтудент Шухова А.\,В. \tline{(подпись)}{1in}
\pagebreak


\begin{center}
\bfseries{\large ЗАДАНИЕ}
\end{center}

кафедры 806 по вычислительной практике: \\
написать сайт и парсер для сбора информации о котах, находящихся в приюте, с сохранением данных в файл, который после загружается на созданный сайт и отображается на нем. 

\vspace*{\fill}
Руководитель практики от института:

\vspace{5pt}
\enquote{\hspace{0.5cm}} \tline{(дата)}{1.5in} \the\year\,г.\hfillКухтичев А.\,A. \tline{(подпись)}{1in}
\pagebreak

%\begin{table}
\center{\large ТАБЕЛЬ ПРОХОЖДЕНИЯ ПРАКТИКИ}
\vspace{0.5cm}

\begin{tabular}{|c|c|c|c|c|c|}
\hline
& Содержание & Место & \multicolumn{2}{c|}{Время работы} & Подпись цехового\\
\cline{4-5}
Дата & \hspace{1cm} проделанной работы \hspace{1cm} & работы & Начало & Конец &  руководителя \\
\hline
29.06.2019 & Получение задания & МАИ & 9:00 & 18:00 & \\
\hline
01.07.2019 & Сбор информации о создании & МАИ & 9:00 & 18:00 & \\
& сайта и о парсере & & & & \\
\hline
02.07.2019 & Создание сайта с выводом текста & МАИ & 9:00 & 18:00 & \\
& и знакомство со scrapy & & & & \\
\hline
03.07.2019 & Знакомство с моделями, джанго- & МАИ & 9:00 & 18:00 & \\
& админка, создание основы парсера & & & & \\
\hline
& Попытка реализации вывода котов на & & & & \\
04.07.2019 & сайте, изучение кода страницы & МАИ & 9:00 & 18:00 &\\
& сайта приюта, написание для него об- & & & & \\
& хода страницы для сбора информации & & & & \\
\hline
05.07.2019 & Реализация вывода котов на сайте, & МАИ & 9:00 & 18:00 & \\
& исправление ошибок при парсинге & & & & \\
\hline
& Знакомство с bootstrap, приведение & & & & \\
06.07.2019 & записи информации в файл json & МАИ & 9:00 & 18:00 & \\
& к корректному виду & & & & \\
\hline
& Улучшение внешнего вида страницы & & & & \\
07.07.2019 & сайта, реализация обхода всех страниц & МАИ & 9:00 & 18:00 & \\
& сайта приюта, финальный json-файл & & & & \\
\hline
09.07.2019 & Попытка реализовать загрузку json- & МАИ & 9:00 & 18:00 & \\
& файлa на сайт с выводом содержимого & & & & \\
\hline
10.07.2019 & Продолжение работы с загрузкой & МАИ & 9:00 & 18:00 & \\
& файла на сайт, но в формате html & & & & \\
\hline
11.07.2019 & Подготовка к сдаче, создание & МАИ & 9:00 & 18:00 & \\
& презентиции & & & & \\
\hline
12.07.2019 & Сдача журнала & МАИ & 9:00 & 18:00 &  \\
\hline
\end{tabular}
%\end{table}

\pagebreak

{\small}
%\input{src/problem}
%\input{src/tabel}
\begin{center}
\bfseries{\large Отзывы цеховых руководителей практики}
\end{center}

\begin{flushleft}
Студенты Шухова А.\,В. и Буркевич В.\,Д. разработали парсер, выполняющий обход сайта приюта и собирающий информацию про котов, с записью данных в json-файл и сайт, который позволяет до- бавлять новых котов, показывать уже веденных и выводить содержащуюся о котах информацию в файле, который был получен путем реализованого ране парсера.

Презентация защищена на комиссии кафедры 806. Работа выполнена в полном объёме. Рекомендую на оценку \enquote{\hspace{2cm}}. Все материалы сданы на кафедру.
\end{flushleft}
\pagebreak

\center{\large МАТЕРИАЛЫ ПО РАЦИОНАЛИЗАТОРСКИМ ПРЕДЛОЖЕНИЯМ}

\begin{flushleft}
{\bfseries Для сайта.} Для улучшения сайта нужно реализовать кнопку, которая позволит загружать на сайт json-файлы, а также файлы других форматов. Так как не удалось реализовать загрузку json-файла и вывод данных происходит в виде таблицы, необходимось в использовании bootstrap'а отпала. По- мимо этого, можно было бы добавить блоки в html, например main, page, для большей конкретики в коде сайта.

{\bfseries Для парсера.} Реализованый парсер выполняет сбор информации только в том случае, когда в ко- де страницы класс с опредленым именем описывает отдельного кота, но если имя класса для каж- дого кота будет индивидуально, то извлечь данные не получится. Реализация обхода такого сайта значительно улучшила бы парсер.
\end{flushleft}
\pagebreak

\center{\large ТЕХНИЧЕСКИЙ ОТЧЕТ ПО ПРАКТИКЕ}

\begin{flushleft}
{\bfseries Архитекрута.} Работа выполена на операционной системе Windows. Для написания сайта был выбран фреймворк Django (язык Python). Парсер также написан на языке Python. Для его реа- лизации использовалась интерактивная оболочка Jupyter Notebook. \\
{\bfseries Описание.} Сайт получился дочтаточно простым и он корректно работает. А код сайта акуратен, легко понимаем. Но сайт не идеальный и требует усовершенствования. Парсер же не имеет боль- шого количества кода и будет вполне понятным для тех, кто захочет подробнее узнать о работе данного парсера. При этом извлекаются все нужные данные в удобном формате.\\
{\bfseries Реализация.} Создание сайта началось с простого проекта, в котором выводился текст. Постепен- но была реализована внутренняя составляющая: views, models и другие файлы, а также джанго- админка. Далее реализация возможности создания котов на сайте, то есть их отображение в базе данных. Вывод всех котов таблицей в html. Парсер был реализован с помощью Scrapy - среда для извлечения данных веб-страниц с открытым кодом. Для обхода сайта и сбора информации приме- нются селекторы. Также создается отдельный класс, который описывает способ обхода сайта, на- зываемый пауком (Spider). В нем описан поиск необходимой информации на сайте с проходом по всем страницам.\\
{\bfseries Тестирование.} Тестирование парсера проводилось в Jupyter Notebook путем запуска обхода сайта, после чего информация из полученного файла сравнивалась с данными с сайта. А тестирование ра- боты сайта проводилось путем взоиможействия с сайтом, то есть попытками добавить кота, загру- зить файл и т.п.\\
{\bfseries Ссылка на GitHub.} https://github.com/AlexandraShukhova/site
\end{flushleft}
\end{document}
